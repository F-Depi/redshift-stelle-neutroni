% Run this command from vim
% :!pdflatex -interaction=batchmode main.tex && evince main.pdf &
% then to run the last command that starts with :! just do :!!

\documentclass[a4paper, titlepage]{article}

\usepackage[T1]{fontenc}
\usepackage[utf8]{inputenc}
\usepackage[italian]{babel}
%\usepackage[hashEnumerators,smartEllipses]{markdown}
\usepackage{mathtools}
\usepackage{siunitx} % Load siunitx first
\usepackage{physics} % Load physics after siunitx
%\usepackage{amsmath} %mathtools loads amsmath too!!
\usepackage{amssymb}
\usepackage[overload]{empheq}
\usepackage{listings}
\usepackage{tabularx}
\usepackage{textcomp}
\usepackage{multirow}
\usepackage{multicol}
\usepackage{booktabs}
\usepackage{graphicx}
\usepackage{floatflt}
\usepackage{epsfig}
\usepackage{pstricks}
\usepackage{subcaption}
\usepackage[labelfont=bf, font=scriptsize]{caption}
\usepackage[italian]{varioref}
\usepackage[suftesi,write]{frontespizio}
\usepackage{xcolor}
\usepackage{caption}
\usepackage{pgfplots}
\usepackage{comment}
\usepackage{bm}            % special bold-math package. usge: \bm{mathsymbol}
\usepackage{array}
\usepackage{lipsum}
\usepackage{csquotes}
\usepackage{biblatex}
%\addbibresource{sample-paper.bib}
\usepackage[colorlinks=true]{hyperref}  % this package should be added after all others.
\pgfplotsset{compat=1.16}
\usepackage[text={15.5cm,23.5cm},centering,heightrounded]{geometry}
\DeclareCaptionType{eq_caption}[Equazione][Elenco delle equazioni]

\definecolor{mygreen}{rgb}{0,0.6,0}
\definecolor{mygray}{rgb}{0.5,0.5,0.5}
\definecolor{mymauve}{rgb}{0.58,0,0.82}

\lstset{ 
  backgroundcolor=\color{white},   % choose the background color; you must add \usepackage{color} or \usepackage{xcolor}; should come as last argument
  basicstyle=\footnotesize,        % the size of the fonts that are used for the code
  breakatwhitespace=false,         % sets if automatic breaks should only happen at whitespace
  breaklines=true,                 % sets automatic line breaking
  captionpos=b,                    % sets the caption-position to bottom
  commentstyle=\color{mygreen},    % comment style
  deletekeywords={...},            % if you want to delete keywords from the given language
  escapeinside={\%*}{*)},          % if you want to add LaTeX within your code
  extendedchars=true,              % lets you use non-ASCII characters; for 8-bits encodings only, does not work with UTF-8
  firstnumber=1,                   % start line enumeration with line 1000
  frame=single,	                 % adds a frame around the code
  keepspaces=true,                 % keeps spaces in text, useful for keeping indentation of code (possibly needs columns=flexible)
  keywordstyle=\color{blue},       % keyword style
  %language=Octave,                % the language of the code
  morekeywords={*,...},            % if you want to add more keywords to the set
  numbers=left,                    % where to put the line-numbers; possible values are (none, left, right)
  numbersep=5pt,                   % how far the line-numbers are from the code
  numberstyle=\tiny\color{mygray}, % the style that is used for the line-numbers
  rulecolor=\color{black},         % if not set, the frame-color may be changed on line-breaks within not-black text (e.g. comments (green here))
  showspaces=false,                % show spaces everywhere adding particular underscores; it overrides 'showstringspaces'
  showstringspaces=false,          % underline spaces within strings only
  showtabs=false,                  % show tabs within strings adding particular underscores
  stepnumber=1,                    % the step between two line-numbers. If it's 1, each line will be numbered
  stringstyle=\color{mymauve},     % string literal style
  tabsize=2,	                   % sets default tabsize to 2 spaces
  title=\lstname                   % show the filename of files included with \lstinputlisting; also try caption instead of title
}

%\begin{frontespizio}
    \Universita{Trento} % CTT
    \Logo{Figures/logo_unitn} % CTT
    \Divisione{Fisica Computazionale} % CTT
    \Corso[Laurea Triennale]{Fisica} % CTT, a meno che non cambi la denominazione del corso
    \Annoaccademico{2023-2024}
    \Titoletto{Progetto Finale} % CTT
    \Titolo{Calcolo redshift nell'emissione di fotoni da una stella di neutroni\\ }
    \Sottotitolo{\today}
    \Candidato[227552]{Federico De Paoli, \textsf {federico.depaoli@studenti.unitn.it}}
    \NRelatore{Docente}{} % CTT
    \Relatore{Prof. Alessandro Roggero} % CTT, a meno che non sia cambiato il Prof.
\end{frontespizio}
\IfFileExists{\jobname-frn.pdf}{}{%
\immediate\write18{pdflatex \jobname-frn}} % ASSOLUTAMENTE CTT, è il comando che materialmente vi genera il frontespizio.

\newpage



\begin{document}

\section{Introduzione}
Studiamo la stabilità delle stelle di neutroni in regime relativistico considerando 3 possibili equazioni di stato per la materia.
Una volta risolte le equazioni è possibile ottenere l'espressione del potenziale gravitazione della stella e calcolare l'effetto sulla radiazione emessa dalla stella.


Viene calcolata la radianza per ogni stella a 3 distanze diverse e la potenza totale di emissione in funzione della distanza dalla stella.
Viene quindi calcolata la temperatura apparente delle 3 stelle più massive in funzione di quella effettiva e poi viene studiata la temperatura apparente in funzione della pressione centrale della stella.

\section{Stabilità}

Le equazioni che descrivono la stabilità di una stella in funzione della massa ($m$) e della pressione ($P$) sono quelle di Tolman-Oppenheimer-Volkoff
\begin{equation}
    \begin{dcases}
        \dv[]{P(r)}{r} = - G \frac{m(r) \epsilon (r)}{r^2 c^2} \left(1 + \frac{P(r)}{\epsilon (r)} \right) \left(\ + \frac{4 \pi r^3 P(r)}{m(r) c^2} \right) \left(1 - \frac{2 G m(r)}{r c^2} \right)^{-1} \\
        \dv[]{m(r)}{r} = 4 \pi r^2 \frac{\epsilon (r)}{c^2} \\
        \dv[]{\Phi(r)}{r} = - \frac{1}{P(r) + \epsilon (r)} \dv[]{P(r)}{r}
    \end{dcases}
    \label{eq:sistema_corr}
\end{equation}

Dove la terza equazione è l'equazione disaccoppiata e descrive il potenziale gravitazionale della stella.
Usiamo 3 diverse densità di energia per la materia della stella (eq. \ref{eq:energia1} viene presa con due coppie di valori diversi di $\Gamma$ e $K$):
\begin{align}
    \epsilon_1 (n) &= a \left( \frac{n}{n_0} \right) ^{\alpha} + b \left( \frac{n}{n_0} \right) ^{\beta} \\
    \epsilon_{2/3} (n) &= \mu c^2n+Kc^2n^\Gamma
    \label{eq:energia1}
\end{align}
\begin{equation}
    \text{con} \quad a = 13.4 \unit{\mega\electronvolt\per\femto\cubic\meter}, \quad
    \alpha = 0.514, \quad
    b = 5.62 \unit{\mega\electronvolt\per\femto\cubic\meter}, \quad
    \beta = 2.436, \quad
    n_0 = 0.16 \unit{\per\femto\cubic\meter} \\
\end{equation}

dove $n$ è la densità numerica, $\mu$ la massa di una singola particella e quindi $\rho = \mu n$ è la densità di massa.


Visto che la densità di energia è in funzione di $\rho$ e le incognite del sistema \ref{eq:sistema_corr} sono $P$ e $m$ possiamo scrivere la densità di energia in funzione di $P$ e $m$ partendo dalla relazione termodinamica

\begin{equation}
    P = - \dv[]{E}{V} \quad \Rightarrow \quad
    \begin{cases}
        P = (\alpha - 1) a \left( \frac{n}{n_0} \right)^{\alpha} (\beta - 1) b \left( \frac{n}{n_0} \right)^{\beta} & \text{per } \epsilon_1 \\
        n = \left( \frac{P}{K(\Gamma - 1) c^2} \right) ^ {1 / \Gamma} & \text{per } \epsilon_{1/2}
    \end{cases}
    \label{eq:eps(r)}
\end{equation}

Nel primo caso non è stato possibile invertire l'equazione per trovare $n$ in funzione di $P$ e $m$ quindi utilizzeremo un metodo numerico per trovare $n$ di volta in volta.


Facciamo le seguenti sostituzioni per rendere le variabili adimensionali e con valori più vicini a 0.

\begin{equation*}
    m=M_0\hat m, \quad 
    r=R_0\hat r, \quad 
    P=P_0\hat P, \quad
    \rho=\rho_0 \hat{\rho}, \quad
    K = \hat{K}\frac{\mu^\Gamma}{\rho_0^{\Gamma-1}},
\end{equation*}

\begin{equation}
    \begin{dcases}
        \dv[]{\hat P}{\hat r} = - \frac{(\hat P + \hat{\epsilon})(\hat m + \hat r^3 \hat P)}{\hat r^2-2\hat m\hat r}\\
        \dv[]{\hat m}{\hat r} = \hat r^2 \hat{\epsilon} \\
        \dv[]{\Phi}{\hat r} = - \frac{1}{\hat P + \hat{\epsilon}} \dv[]{\hat P}{\hat r}
    \end{dcases}
    \label{eq:sistema_adim}
\end{equation}

Otteniamo il sistema \ref{eq:sistema_adim} dove, grazie alle equazioni in \ref{eq:eps(r)}, $\hat m$, $\hat P$ e $\hat{\epsilon}$ sono funzioni di $\hat r$. Come valori delle costanti sono stati usati

\begin{equation}
    M_0 = 12.655756 M_\odot \quad R_0 = 20.06145 \unit{\kilo\meter} \quad \epsilon = P_0 = \rho_0 c^2 = 150.174 \frac{\unit{\mega\electronvolt}}{c^2 \unit{\femto\meter}}
\end{equation}

Per il potenziale gravitazionale $\Phi$ si può inoltre trovare una soluzione analitica all'esterno della stella che possiamo mettere in forma adimensionale (eq. \ref{eq:Phi_ext}), dove $\hat{M}$ e $\hat R$ sono rispettivamente la massa totale e il raggio della stella in forma adimensionale.

\begin{equation}
    \Phi_\text{ext} (r) = \frac{1}{2} \log(1 - \frac{2 G M}{r c^2})
    \implies \Phi_\text{ext} (\hat r) = \frac{1}{2} \log(1 - \frac{2 \hat{M}}{\hat r}) \quad \quad \hat r \geq \hat{R}
    \label{eq:Phi_ext}
\end{equation}

\end{document}


