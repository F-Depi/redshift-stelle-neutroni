\documentclass[a4paper, titlepage]{article}

\input{packages}
\begin{frontespizio}
    \Universita{Trento} % CTT
    \Logo{Figures/logo_unitn} % CTT
    \Divisione{Fisica Computazionale} % CTT
    \Corso[Laurea Triennale]{Fisica} % CTT, a meno che non cambi la denominazione del corso
    \Annoaccademico{2023-2024}
    \Titoletto{Progetto Finale} % CTT
    \Titolo{Calcolo redshift nell'emissione di fotoni da una stella di neutroni\\ }
    \Sottotitolo{\today}
    \Candidato[227552]{Federico De Paoli, \textsf {federico.depaoli@studenti.unitn.it}}
    \NRelatore{Docente}{} % CTT
    \Relatore{Prof. Alessandro Roggero} % CTT, a meno che non sia cambiato il Prof.
\end{frontespizio}
\IfFileExists{\jobname-frn.pdf}{}{%
\immediate\write18{pdflatex \jobname-frn}} % ASSOLUTAMENTE CTT, è il comando che materialmente vi genera il frontespizio.

\newpage



\begin{document}

\section{Introduzione}
Studio readshift di stelle di neutroni. Le euqazioni che descrivono la stabilità di una stella di neutroni sono

\begin{equation}
    \begin{dcases}
        \dv[]{P(r)}{r} = - G \frac{m(r) \epsilon (r)}{r^2 c^2} \left(1 + \frac{P(r)}{\epsilon (r)} \right) \left(\ + \frac{4 \pi r^3 P(r)}{m(r) c^2} \right) \left(1 - \frac{2 G m(r)}{r c^2} \right)^{-1} \\
        \dv[]{m(r)}{r} = 4 \pi r^2 \frac{\epsilon (r)}{c^2}
    \end{dcases}
    \label{eq:sistema_corr}
\end{equation}

Come densità di energia interna usiamo

\begin{equation}
    \epsilon (\rho) = a \rho ^{\alpha} + b \rho ^{\beta}
\end{equation}

Prendiamo \ref{eq:sistema_corr} e lo adimensioniamo con

\begin{equation}
    \begin{dcases}
        \dv[]{\hat{P}}{\hat{r}}=-\frac{(\hat{P}+\hat{\epsilon})(\hat{m}+\hat{r}^3\hat{P})}{\hat{r}^2-2\hat{m}\hat{r}}\\
        \dv[]{\hat{m}}{\hat{r}}= \hat{r}^2 \hat{\epsilon}
    \end{dcases}
    \label{eq:sistema_adim}
\end{equation}

otteniamo \ref{eq:sistema_adim} con


\end{document}
